% header
\documentclass[10pt,a4paper]{article}

\usepackage[utf8]{inputenc}
\usepackage{hyperref}
\usepackage{amssymb}
\usepackage{ngerman}
\usepackage{listings}
\usepackage{amsmath}
\usepackage{dsfont}


\newcommand\iv{\mathrel{\overset{\makebox[0pt]{\mbox{\normalfont\tiny\sffamily IV}}}{=}}}

% the document
\begin{document}

% create the title
% Please replace the data in brackets [] with actual data.
\title{Abgabe - Übungsblatt [$10$]\\
\small{Angewandte Mathematik: Numerik}}
\author{ [Felix Lehmann] \and [Markus Menke]}
\date{\today}
\maketitle

\section*{Aufgabe 1}
Zuerst zeigen wir per Induktion über i, dass $A^{i} = V \cdot \Lambda^{i} \cdot V^{-1}$\\
IV: Die Aussage gelte für ein beliebiges aber festes n.\\
IA: $n = 1$:\\
Trivial, die Zerlegung $A = V \cdot \Lambda \cdot V^{-1}$ existiert nach Aufgabenstellung\\
IS: $n \mapsto n+1$:\\
$A^{n+1} = A^{n} \cdot A \iv V \cdot \Lambda^{n} \cdot V^{-1} \cdot A = V \cdot \Lambda^{n} \cdot V^{-1} \cdot V \cdot \Lambda \cdot V^{-1}\\
= V \cdot \Lambda^{n} \cdot E \cdot \Lambda \cdot V^{-1} = V \cdot \Lambda^{n+1} \cdot V^{-1}$\\
Dies galt zu zeigen.\\

$g(A):=  \sum_{n=0}^{\infty} c_{i} \cdot A^{i} = \sum_{n=0}^{\infty} c_{i} \cdot V \cdot \Lambda^{i} \cdot V^{-1}\\
= V \cdot (\sum_{n=0}^{\infty} c_{i}\cdot\Lambda^{i}) \cdot V^{-1} = V \cdot diag(f(\lambda_{1}),...,f(\lambda_{m})) \cdot V^{-1} $\\
Dies galt zu zeigen.\\
 
\section*{Aufgabe 2}
And some more text \ldots

\end{document}
