% header
\documentclass[10pt,a4paper]{article}

\usepackage[utf8]{inputenc}
\usepackage{hyperref}
\usepackage{amssymb}
\usepackage{ngerman}
\usepackage{listings}
\usepackage{amsmath}
\usepackage{dsfont}

% the document
\begin{document}

% create the title
% Please replace the data in brackets [] with actual data.
\title{Abgabe - Übungsblatt [11]\\
\small{Angewandte Mathematik: Numerik}}
\author{ [Felix Lehmann] \and [Markus Menke]}
\date{\today}
\maketitle

\section*{Aufgabe 1}
Here comes your text \ldots

\section*{Aufgabe 2}
Wir interpretieren die Übergänge als Matrizenmultiplikation:\\
$\begin{pmatrix}
    5 && -1\\
    6 && 12\\
\end{pmatrix} \dot{}
\begin{pmatrix}
    a_{n}\\
    b_{n}\\
\end{pmatrix} = 
\begin{pmatrix}
    a_{n+1}\\
    b_{n+1}\\
\end{pmatrix}$\\
Sei A = $\begin{pmatrix}
    5 && -1\\
    6 && 12\\
\end{pmatrix}$\\
Wir berechnen die Eigenwerte:\\
$det\begin{pmatrix}
    5-\lambda && -1\\
    6 && 12-\lambda\\
\end{pmatrix}=\lambda^2 -17\lambda + 66 \rightarrow \lambda_1 = 11, \lambda_2 = 6$\\
Die Eigenvektoren lauten:\\
$\begin{pmatrix}
    -1\\
    6\\
\end{pmatrix}\\ 
\begin{pmatrix}
    -1\\
    1\\
\end{pmatrix}$\\
Dadurch ergibt sich eine Eigenwertzerlegung:\\
$A = V \dot{} \Lambda \dot V^{-1} = \begin{pmatrix}
    -1 && -1\\
    6 && 1\\
\end{pmatrix} \dot{} \begin{pmatrix}
    11 && 0\\
    0 && 6\\
\end{pmatrix} \dot{}\begin{pmatrix}
    1/5 && 1/5\\
    -6/5 && 1/5\\
\end{pmatrix} $\\
Jetzt verwenden wir das Ergebnis von Blatt 10 Aufgabe 1:\\
$A^{n} = V * diag(11^n, 6^n) * V^{-1} = \begin{pmatrix}
    -11^n/5-6^{n+1}/5 && -11^n/5-6^n/5\\
    6/5*11^n-6^{n+1}/5 && 6/5*11^n+6^n/5\\
\end{pmatrix}
\rightarrow A^{n}*\begin{pmatrix}
    2\\
    4\\
\end{pmatrix}=\begin{pmatrix}
    2*(-11^n/5-6^{n+1}/5)+4(-11^n/5-6^n/5)\\
    2*(6*11^n/5-6^{n+1}/5)+4(6*11^n/5+6^n/5)\\
\end{pmatrix}$\\
Die explizite Darstellung der Folgen sind die Zeilen dieses Vektors (Zeile 1: Folge a, Zeile 2: Folge b)\\

\section*{Aufgabe 3}
And even more text \ldots

\section*{Aufgabe 4}

\subsection*{a)}
\begin{align*}
f &= \begin{pmatrix}
    (R + r \cos \varphi) \cos \theta \\
    (R + r \cos \varphi) \sin \theta \\
    r \sin \varphi
\end{pmatrix}\\
\frac{\delta}{\delta\theta}f &= \begin{pmatrix}
    (R + r \cos \varphi)  ( -sin \theta) \\
    (R + r \cos \varphi)  (\cos \theta) \\
    0
\end{pmatrix}\\
\frac{\delta}{\delta\varphi}f &= \begin{pmatrix}
    - r \cos \theta \sin \varphi\\
    - r \sin \theta \sin \varphi\\
    r \cos \varphi
\end{pmatrix}\\
f' &= \begin{pmatrix}
    (R + r \cos \varphi) ( -sin \theta) & - r \cos \theta \sin \varphi \\
    (R + r \cos \varphi) (\cos \theta) & - r \sin \theta \sin \varphi\\
    0 & r \cos \varphi
\end{pmatrix}
\end{align*}

\subsection*{b)}
\begin{align*}
(f\circ\gamma)' &= f'(\gamma(t)) \circ \gamma'(t)\\
&= \begin{pmatrix}
    (R + r \cos (bt^2)) ( -\sin (at^2)) & - r \cos (at^2) \sin (bt^2)\\
    (R + r \cos (bt^2)) (\cos (at^2)) & - r \sin (at^2) \sin (bt^2)\\
    0 & r \cos (bt^2)
\end{pmatrix} \circ \begin{pmatrix} 
    2at\\
    2bt\\
\end{pmatrix}\\
&= \begin{pmatrix}
    2at(R + r \cos (bt^2)) ( -\sin (at^2)) + (- 2bt r \cos (at^2) \sin (bt^2))\\
    2at(R + r \cos (bt^2)) (\cos (at^2)) + (-2bt r \sin (at^2) \sin (bt^2))\\
    2bt r \cos (bt^2)
\end{pmatrix}\\
&= \begin{pmatrix}
    2t (a(R + r \cos (bt^2)) ( -\sin (at^2)) + (- b r \cos (at^2) \sin (bt^2)))\\
    2t (a(R + r \cos (bt^2)) (\cos (at^2)) + (-b r \sin (at^2) \sin (bt^2)))\\
    2btr \cos (bt^2)
\end{pmatrix}\\
\end{align*}
\end{document}
