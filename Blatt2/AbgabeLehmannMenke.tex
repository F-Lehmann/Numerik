% header
\documentclass[10pt,a4paper]{article}

\usepackage[utf8]{inputenc}
\usepackage{hyperref}
\usepackage{amssymb}
\usepackage{ngerman}
\usepackage{amsmath}
\usepackage{mathtools}

% the document
\begin{document}

% create the title
% Please replace the data in brackets [] with actual data.
\title{Abgabe - Übungsblatt [$2$]\\
\small{Angewandte Mathematik: Numerik}}
\author{ [Felix Lehmann] \and [Markus Menke]}
\date{\today}
\maketitle

\section*{Aufgabe 1}
$B = \begin{pmatrix}
    -2 &  1 & -7\\
    -7 & -1 &  1\\
     0 &  6 &  5
    \end{pmatrix}$

\subsection*{a)}
$||B||_1 = 13$\\
$||B||_\infty = 11$\\
$||B||_F = 2\sqrt{41}$

\subsection*{b)}
1. Zu zeigen: $||x||_W \geq 0, ||x||_W = 0$ gdw. $ x = 0$\\

Der Beweis erfolgt in drei Schritten.\\
Wir zeigen zunächst, dass $||x||_W \geq 0$:
Laut Definition ist $||x||_W \coloneqq ||W*x||$. Da $||*||$ eine Norm ist, muss $||x|| \geq 0$ gelten. Somit ist gilt auch $||W*x|| \geq 0$. Damit ist die erste Bedingung erfüllt.\\

Als nächsten Schritt zeigen wir die Hinrichtung der Äquivalenz.\\ Hierbei nutzen wir die absorbierende Eigenschaft von 0 und die Eigenschaft $||x|| = 0 \Rightarrow x = 0$ der Norm $||*||$:

$x = 0 \Rightarrow ||x||_W = ||0||_W \coloneqq ||W*0|| = ||0|| = 0$\\

Als letzten Schritt zeigen wir die Rückrichtung der Äquivalenz.\\
$||x||_W = 0 \Rightarrow ||x||_W \coloneqq ||W*x||$\\
Sei $y$ das Ergebnis von $W*x$, dann ist $y_i \coloneqq \sum_{j=1}^m w_ij * x_j$.\\
Da $W$ invertierbar ist, sind auch alle Zeilen von $W$ linear unabhängig. Somit ist der Vektor $x=0$ eindeutig bestimmt. Damit sind alle Bedingungen für Schritt 1 erfüllt.\\

2. Zu zeigen: $||\alpha x||_W = |\alpha|*||x||_W$\\

$||\alpha*x||_W \coloneqq || W * ( \alpha * x )||$\\
Sei $y$ das Ergebnis von $W*(\alpha * x)$, dann gilt aufgrund der Kommutativität von $*$ in $\mathbb{C}$:\\ 
$y_i = \sum_{j=1}^m w_ij * (\alpha * x_j) = \alpha*\sum_{j=1}^{w_i}j*x_j$.\\ 
Somit ist $w*(\alpha * x) = \alpha *(w*x)$\\

Da $||*||$ eine Norm ist, gilt $||\alpha*A|| = |\alpha|*||A||$.\\
Sei $A\coloneqq W*x$, so gilt aufgrund der Eigenschaft der Norm $||*||$ auch $||\alpha*(W*x)|| = |\alpha|*||W*x|| =  = |\alpha|*||x||_W$. Dies galt zu zeigen.\\

3. Zu zeigen: $||x+y||_W \leq ||x||_W +||y||_W$\\

Sei $y'$ das Ergebnis von $W*(x+y)$, dann ist $y'_i = \sum_{j=1}^m w_ij * ( x_j + y_j ) = \sum_{j=1}^m w_ij * x_j + \sum_{j=1}^m w_ij * y_j$ somit ist $W*(x+y) = Wx + Wy$. Aufgrund der Eigenschaft der Norm $||*||$ gilt dann auch $||Wx + Wy|| \leq ||Wx|| + ||Wy||$.

Da alle 3 Eigenschaften für eine Norm erfüllt sind, bildet $||x||_W$ eine Norm.

\subsection*{c)}
Eine Norm muss drei Eigenschaften erfüllen:\\
$||A|| = 0 \Rightarrow A = 0$\\
$||\alpha*A|| = |\alpha|*||A||$\\
$||A+B|| \leq ||A|| + ||B||$\\
Die gegebene Abbildung erfüllt diese, und ist damit eine Norm.

\section*{Aufgabe 2}
\subsection*{a)}
Ist für $m=1$ immer gleich.
Für $m \geq 2$ das größte element allein ist immer $\leq$ als das größte element $+ x\epsilon R_0^+$.

\subsection*{b)}
Ist für $m=1$ immer gleich.

\subsection*{c)}
Ist für $m,n=1$ immer gleich.

\subsection*{d)}
Ist für $m,n=1$ immer gleich.

\section*{Aufgabe 3}
\subsection*{a)}
$ v * ((I - \frac{2}{v*v} \cdot v \cdot v^*)\cdot w) = -v * w$\\
$v^*$ wird ist nicht eindeutig definiert, Aufgabe daher nicht lösbar

\section*{Aufgabe 4}
\subsection*{a)}
$a_1 \cdot a_2 = 0$\\
$a_1 \cdot a_3 = 0$\\
$a_1 \cdot a_4 = 0$\\
$a_2 \cdot a_3 = 0$\\
$a_2 \cdot a_4 = 0$\\
$a_3 \cdot a_4 = 0$\\
damit ist $A$ paarweise orthogonal
$c_1 \cdot c_2 = -22i$\\
$c_1 \cdot c_3 = 22i$\\
$c_2 \cdot c_3 = 48+46i$\\
damit ist $C$ nicht paarweise orthogonal

\subsection*{b)}
$$A^* \cdot A = \begin{pmatrix}
    0&1&1&0\\
    1&0&0&1\\
    1&0&1&0\\
    0&1&0&1\\
    1&0&0&1\\
    0&1&1&0\\
    0&1&0&1\\
    1&0&1&0
\end{pmatrix}$$
$(A^* \cdot A)^{-1}$ ist nicht definiert
$$(A^* \cdot A)^{-1} \cdot A^* = \frac{1}{4}\begin{pmatrix}
    0&1&1&0&1&0&0&-1\\
    1&0&0&-1&0&1&1&0\\
    1&0&-1&0&0&-1&0&-1\\
    0&1&0&1&-1&0&1&0
\end{pmatrix}$$

\end{document}
