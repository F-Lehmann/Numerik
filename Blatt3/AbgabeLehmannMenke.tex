% header
\documentclass[10pt,a4paper]{article}

\usepackage[utf8]{inputenc}
\usepackage{hyperref}
\usepackage{amssymb}
\usepackage{ngerman}
\usepackage{listings}
\usepackage{amsmath}

\DeclareMathOperator*{\argmin}{arg\,min}

% the document
\begin{document}

% create the title
% Please replace the data in brackets [] with actual data.
\title{Abgabe - Übungsblatt [$3$]\\
\small{Angewandte Mathematik: Numerik}}
\author{ [Felix Lehmann] \and [Markus Menke]}
\date{\today}
\maketitle

\section*{Aufgabe 1}
A ist gleich der Vandermonde-Matrix\\
$A = \begin{pmatrix}
 1 & -1 & 1 \\
 1 & 0  & 0 \\
 1 & 1  & 1 \\
 1 & 2  & 4 \\
\end{pmatrix}\\
v = \begin{pmatrix}
    1 \\
    0 \\
    2 \\
    4 \\
\end{pmatrix}$

So ergibt sich das folgende lineare Ausgleichsproblem:

$\begin{pmatrix}
    c \\
    b \\
    a \\
\end{pmatrix} = \argmin_y ||v-Ay||_2$

Nach Skript Satz 2.2 suchen wir nun nach einer Lösung der Gausschen Normalengleichung: \\

$A* Ax = A* v$

Setzen wir unsere Werte ein, erhalten wir folgende Gleichung:\\

$\begin{pmatrix}
    1 & 1 & 1 & 1\\
    -1 & 0  & 1 & 2\\
    1 & 0  & 1 & 4 \\
\end{pmatrix}
\begin{pmatrix}
    1 & -1 & 1 \\
    1 & 0  & 0 \\
    1 & 1  & 1 \\
    1 & 2  & 4 \\
   \end{pmatrix} 
   \begin{pmatrix}
    c \\
    b \\
    a \\
\end{pmatrix} = 
\begin{pmatrix}
    1 & 1 & 1 & 1\\
    -1 & 0  & 1 & 2\\
    1 & 0  & 1 & 4 \\
\end{pmatrix} 
\begin{pmatrix}
    1 \\
    0 \\
    2 \\
    4 \\
\end{pmatrix}$  \\


Zuerst lösen wir die Matrixprodukte auf

$
\leftrightarrow
\begin{pmatrix}
    4 & 2 & 6\\
    2 & 6 & 8\\
    6 & 8 & 18
\end{pmatrix}
\begin{pmatrix}
    c \\
    b \\
    a \\
\end{pmatrix} 
=
\begin{pmatrix}
    7 \\
    9 \\
    19 \\
\end{pmatrix} 
$ 

Wir lösen das LGS mit dem Gauss-Algorithmus:\\

$
\left(
\begin{array}{ccc | c}
    4 & 2 & 6 & 7\\
    2 & 6 & 8 & 9\\
    6 & 8 & 18 & 19
\end{array}
\right) \leftrightarrow
\left(
\begin{array}{ccc | c}
    4 & 2 & 6 & 7\\
    4 & 12 & 16 & 18\\
    12 & 16 & 36 & 38
\end{array}
\right)  \leftrightarrow
\left(
\begin{array}{ccc | c}
    4 & 2 & 6 & 7\\
    0 & 10 & 10 & 11\\
    0 & 10 & 18 & 17
\end{array}
\right)  
\\\leftrightarrow
\left(
\begin{array}{ccc | c}
    4 & 2 & 6 & 7\\
    0 & 10 & 10 & 11\\
    0 & 0 & 8 & 16
\end{array}
\right) \rightarrow a = 3/4, b = 7/20, c = 9/20\\
$

\section*{Aufgabe 2}
Nope\ldots

\section*{Aufgabe 3}
a)
\lstinputlisting[language=Python, breaklines=true, frame=single,linerange={6-6,14-23}]{Framework.py}
b)
\lstinputlisting[language=Python, breaklines=true, frame=single, linerange={26-26,36-49}]{Framework.py}

\end{document}
