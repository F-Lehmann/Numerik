% header
\documentclass[10pt,a4paper]{article}

\usepackage[utf8]{inputenc}
\usepackage{hyperref}
\usepackage{amssymb}
\usepackage{ngerman}
\usepackage{listings}
\usepackage{amsmath}
\usepackage{dsfont}

% the document
\begin{document}

% create the title
% Please replace the data in brackets [] with actual data.
\title{Abgabe - Übungsblatt [$4$]\\
\small{Angewandte Mathematik: Numerik}}
\author{ [Felix Lehmann] \and [Markus Menke]}
\date{\today}
\maketitle

\section*{Aufgabe 1}
\subsection{Teilaufgabe a}
\begin{enumerate}
    \item Berechne  $B = A^TA$\\
    
    $B = \begin{pmatrix}
        1 & 2 & 0\\
        0 & 1 & 1
    \end{pmatrix}
    \begin{pmatrix}
        1 & 0\\
        2 & 1\\
        0 & 1
    \end{pmatrix} =
    \begin{pmatrix}
        5 & 2\\
        2 & 2
    \end{pmatrix}\\
    $
    \item Eigenwerte berechnen\\
    
    $det( B - \lambda E) \overset{!}{=} 0$\\
    Wir erhalten das charakteristische Polynom: \\
    $\lambda^2 - 7\lambda +6$\\
    Die Eigenwerte sind die Nullstellen des Polynoms:\\
    $\lambda_1 = 6$\\
    $\lambda_2 = 1$\\

    \item Berechne V\\
    
    Eigenvektor von $\lambda_2$:\\
    $(B-\lambda_2) * v_2 \overset{!}{=} 0$\\
    Wir lösen das LGS und erhalten unseren Eigenvektor\\
    $v_2 = t_2 * \begin{pmatrix}
        -1/2\\
        1
    \end{pmatrix}, t_2 \in \mathds{R}/\{0\}\\
    $
    Wir normieren den Vektor\\
    $v_2 = \begin{pmatrix}
        -1/\sqrt{5}\\
        2/\sqrt{5}
    \end{pmatrix}$

    Eigenvektor von $\lambda_1$:\\
    $(B-\lambda_1) * v_1 \overset{!}{=} 0$\\
    Wir lösen das LGS und erhalten unseren Eigenvektor\\
    $v_1 = t_1 * \begin{pmatrix}
        2\\
        1
    \end{pmatrix}, t_1 \in \mathds{R}/\{0\}\\
    $
    Wir normieren den Vektor\\
    $v_1 = \begin{pmatrix}
        2/\sqrt{5}\\
        1/\sqrt{5}
    \end{pmatrix}$

    Somit ist $V = [v_1,v_2]$ und es gilt $V*V^T = E$

    \item Bilde die Diagonalmatrix $\Sigma$\\
    
    $\Sigma = \begin{pmatrix}
        \sqrt{6} & 0\\
        0 &  1
    \end{pmatrix}$

    \item Berechne U\\
    $u_2 = \dfrac{1}{\sqrt{\lambda_2}}Av_2 = \begin{pmatrix}
        1 & 0\\
        2 & 1\\
        0 & 1
    \end{pmatrix} * \begin{pmatrix}
        -1/\sqrt{5}\\
        2/\sqrt{5}
    \end{pmatrix} =
    \begin{pmatrix}
        \dfrac{-\sqrt{5}}{5}\\
        0\\
        \dfrac{2\sqrt{5}}{5}
    \end{pmatrix}$\\
    $u_1 = \dfrac{1}{\sqrt{\lambda_1}}Av_1 = \dfrac{1}{\sqrt{6}} \begin{pmatrix}
        1 & 0\\
        2 & 1\\
        0 & 1
    \end{pmatrix} * \begin{pmatrix}
        2/\sqrt{5}\\
        1/\sqrt{5}
    \end{pmatrix} =
    \begin{pmatrix}
        \dfrac{\sqrt{30}}{15}\\
        \dfrac{\sqrt{30}}{6}\\
        \dfrac{\sqrt{30}}{30}
    \end{pmatrix}$\\

    Dadurch ergibt sich folgende Zerlegung:

    $A =     \begin{pmatrix}
        \dfrac{\sqrt{30}}{15} & \dfrac{-\sqrt{5}}{5} & *\\
        \dfrac{\sqrt{30}}{6}  & 0 & *\\ 
        \dfrac{\sqrt{30}}{30} & \dfrac{2\sqrt{5}}{5} & *
    \end{pmatrix} * \begin{pmatrix}
        \sqrt{6} & 0\\
       0  & 1 \\ 
        0 & 0
    \end{pmatrix} * \begin{pmatrix}
        2/\sqrt{5} & -1/\sqrt{5}\\
        1/\sqrt{5}  & 2/\sqrt{5}
    \end{pmatrix} $
    

\end{enumerate}
\subsection{Teilaufgabe b}
    Die *-Einträge in U fallen in der Multiplikation mit $\Sigma$ aufgrund der 0-Einträge weg.

\section*{Aufgabe 2}
\lstinputlisting[language=Python, breaklines=true, frame=single]{FrameworkSVD.py}

\section*{Aufgabe 3}
Here comes your text \ldots

\section*{Aufgabe 4}
\lstinputlisting[language=Python, breaklines=true, frame=single]{FrameworkImage.py}
Im stark komprimierten Bild verbleiben Intänsitäts werte der Zeilen und Spalten.
Da \glqq Stoff\grqq{} hauptsächlich horizontale und vertikale Bestandteile hat, lässt sich dies so besser komprimieren.

\end{document}
