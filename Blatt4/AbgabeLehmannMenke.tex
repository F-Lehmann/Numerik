% header
\documentclass[10pt,a4paper]{article}

\usepackage[utf8]{inputenc}
\usepackage{hyperref}
\usepackage{amssymb}
\usepackage{ngerman}
\usepackage{listings}
\usepackage{amsmath}

% the document
\begin{document}

% create the title
% Please replace the data in brackets [] with actual data.
\title{Abgabe - Übungsblatt [$4$]\\
\small{Angewandte Mathematik: Numerik}}
\author{ [Felix Lehmann] \and [Markus Menke]}
\date{\today}
\maketitle

\section*{Aufgabe 1}
Here comes your text \ldots

\section*{Aufgabe 2}
\lstinputlisting[language=Python, breaklines=true, frame=single]{FrameworkSVD.py}

\section*{Aufgabe 3}
Here comes your text \ldots

\section*{Aufgabe 4}
\lstinputlisting[language=Python, breaklines=true, frame=single]{FrameworkImage.py}
Im stark komprimierten Bild verbleiben Intänsitäts werte der Zeilen und Spalten.
Da \glqq Stoff\grqq{} hauptsächlich horizontale und vertikale Bestandteile hat, lässt sich dies so besser komprimieren.

\end{document}
