% header
\documentclass[10pt,a4paper]{article}

\usepackage[utf8]{inputenc}
\usepackage{hyperref}
\usepackage{amssymb}
\usepackage{ngerman}
\usepackage{listings}
\usepackage{amsmath}
\usepackage{dsfont}

% the document
\begin{document}

% create the title
% Please replace the data in brackets [] with actual data.
\title{Abgabe - Übungsblatt [$5$]\\
\small{Angewandte Mathematik: Numerik}}
\author{ [Felix Lehmann] \and [Markus Menke]}
\date{\today}
\maketitle

\section*{Aufgabe 1}
Here comes your text \ldots

\section*{Aufgabe 2}
QR Zerlegung:\\

$A=\begin{pmatrix}
        -2 & 2 & 3  \\
        2  & 3 & 1  \\
        1  & 4 & -2
    \end{pmatrix}$

$q_1' = a_1 = \begin{pmatrix}
        -2 \\
        2  \\
        1
    \end{pmatrix}=\begin{pmatrix}
        -2 \\
        2  \\
        1
    \end{pmatrix}$

$r_{11}=||q_1'||=\sqrt{-2^2 +2^2 +1^2} = \sqrt{9} = 3$

$q_1=\frac{1}{||q_1'||}*q_1' = \frac{1}{3}*\begin{pmatrix}
        -2 \\
        2  \\
        1
    \end{pmatrix} = \begin{pmatrix}
        -\frac{2}{3} \\
        \frac{2}{3}  \\
        \frac{1}{3}
    \end{pmatrix}$

$r_{12} = q^T_1 * a_2 = \begin{pmatrix}
        -\frac{2}{3} & \frac{2}{3} & \frac{1}{3}
    \end{pmatrix} \times \begin{pmatrix}
        2 \\
        3 \\
        4
    \end{pmatrix}=2$

$q_2' = a_2-r_{12}*q_1 = \begin{pmatrix}
        2 \\
        3 \\
        4
    \end{pmatrix} - 2*\begin{pmatrix}
        -\frac{2}{3} \\
        \frac{2}{3}  \\
        \frac{1}{3}
    \end{pmatrix}$

$r_{22}=||q_2'||=\sqrt{\frac{10}{3}^2 + \frac{5}{3}^2 + \frac{10}{3}^2}=\sqrt{25}=5$

$q_2 = \frac{1}{||q_2'||} * q_2' = 1/5 * \begin{pmatrix}
        \frac{10}{3} \\
        \frac{5}{3}  \\
        \frac{10}{3}
    \end{pmatrix}=\begin{pmatrix}
        \frac{2}{3} \\
        \frac{1}{3} \\
        \frac{2}{3}
    \end{pmatrix}$

$r_{13} = q^T_1 * a_3 = \begin{pmatrix}
        -\frac{2}{3} & \frac{2}{3} & \frac{1}{3}
    \end{pmatrix} \times \begin{pmatrix}
        3 \\
        1 \\
        -2
    \end{pmatrix}=-2$

$r_{23} = q^T_2 * a_3 = \begin{pmatrix}
        \frac{2}{3} & \frac{1}{3} & \frac{2}{3}
    \end{pmatrix} \times \begin{pmatrix}
        3 \\
        1 \\
        -2
    \end{pmatrix}=1$

$q_3'=a_3-r_{13}*q_1-r_{23}*q_2=\begin{pmatrix}
        3 \\
        1 \\
        -2
    \end{pmatrix} + 2*\begin{pmatrix}
        -\frac{2}{3} \\
        \frac{2}{3} \\
        \frac{1}{3}
    \end{pmatrix} - 1*\begin{pmatrix}
        \frac{2}{3} \\
        \frac{1}{3} \\
        \frac{2}{3}
    \end{pmatrix}=\begin{pmatrix}
        1\\
        2\\
        -2
    \end{pmatrix}$

$r_{33}=||q_3'||=\sqrt{1^2+2^2+(-2)^2}=\sqrt{9}=3$

$q_3 = \frac{1}{||q_3'||}*q_3'=\frac{1}{3}*\begin{pmatrix}
    1\\
    2\\
    -2
\end{pmatrix}=\begin{pmatrix}
    \frac{1}{3}  \\
    \frac{2}{3}  \\
    -\frac{2}{3}
\end{pmatrix}$

$Q=\begin{pmatrix}
    q_1 & q_2 & q_3
\end{pmatrix}=\begin{pmatrix}
        -\frac{2}{3} & \frac{2}{3} & \frac{1}{3}  \\
        \frac{2}{3}  & \frac{1}{3} & \frac{2}{3}  \\
        \frac{1}{3}  & \frac{2}{3} & -\frac{2}{3}
    \end{pmatrix}$

$R=\begin{pmatrix}
    r_{11}&r_{12}&r_{13}\\
    0&r_{22}&r_{23}\\
    0&0&r_{33}\\
\end{pmatrix}=\begin{pmatrix}
        3 & 2 & -2 \\
        0 & 5 & 1  \\
        0 & 0 & 3
    \end{pmatrix}$


\section*{Aufgabe 3}
$QRx = b <=> Rx = Q^*b$, da Q unitär\\
$Q^*b = \frac{1}{2}\begin{pmatrix}
    1 & -1 & 0 & -i & i\\
    0 & 0 & 2 & 0 & 0\\
    1 & 1 & 0 & -1 & -1\\
\end{pmatrix}
\begin{pmatrix}
    2\\
    2i\\
    6\\
    2i\\
    8
\end{pmatrix} = 
\begin{pmatrix}
    1+3i\\
    6\\
    -2
\end{pmatrix}$\\
x ergibt sich nun durch Rückwärtssubstitution\\
$x_3 = -2/2 = 1\\
x_2 = (6 - \sum_{j=3}^{3} R_{2j}x_j)/1 = 7\\
x_1 = (1+3i - \sum_{j=2}^{3} R_{1j}x_j) / 3 = \frac{1}{3}\\$
Somit ist $ = \begin{pmatrix}
    \frac{1}{3}\\
    7\\
    -1\\
\end{pmatrix}$

\section*{Aufgabe 4}
\lstinputlisting[language=Python, breaklines=true, frame=single]{LeastSquares.py}

\end{document}
