% header
\documentclass[10pt,a4paper]{article}

\usepackage[utf8]{inputenc}
\usepackage{hyperref}
\usepackage{amssymb}
\usepackage{ngerman}
\usepackage{listings}
\usepackage{amsmath}
\usepackage{dsfont}

% the document
\begin{document}

% create the title
% Please replace the data in brackets [] with actual data.
\title{Abgabe - Übungsblatt [$7$]\\
\small{Angewandte Mathematik: Numerik}}
\author{ [Felix Lehmann] \and [Markus Menke]}
\date{\today}
\maketitle

\section*{Aufgabe 1}
\subsection*{a)}
\subsection*{exp(x)}
$K_{abs} = |exp(x)|$\\
$K_{rel} = |\frac{exp(x) * x}{exp(x)}| = |x| $\\
Schlecht konditioniert für $x \gg 1$.\\

\subsection*{ln(x)}
$K_{abs} = |\frac{1}{x}|$\\
$K_{rel} = |\frac{\frac{1}{x} * x}{ln(x)}| = |\frac{1}{ln(x)}| $\\
Schlecht konditioniert für $x$ nahe 0.\\

\subsection*{b)}
\subsection*{Absolute Konditionszahl}
Aufgrund der Kettenregel gilt:\\
$K_{abs}(f(g(x))) = |f'(g(x)) * g'(x)| = |f'(g(x))| * |g'(x)| = K_{abs}(f) * K_{abs}(g)$\\

\subsection*{Relative Konditionszahl}
Wir berechnen beide Konditionszahlen.\\
$K_{rel}(f) = |\frac{f'(g(x)) * g(x)}{f(g(x))}|$\\
$K_{rel}(g) = |\frac{g'(x)*x}{g(x)}|$\\
$K_{rel}(g) * K_{rel}(f) = |\frac{f'(g(x))*g'(x)*x}{f(g(x))}|$\\
Hier lässt sich die Kettenregel rückwärts anwenden.
$= K_{rel}(f(g))$


\section*{Aufgabe 2}
\subsection*{a)}
Es gibt $2^{53-24}-1=2^{29}-1=536870911$ Zahlen z mit doppelter Genauigkeit ziwschen zwei aufeinanderfolgenden Zahlen x und y einfacher Genauigkeit.
\subsection*{b)}
Die kleinste Natürliche Zahl die sich mit einefacher Genauigkeit nicht ohne Rundungsfehler darstellen lässt ist $2^{24}+1 = 16777217$ da nur 24 Mantissen Bits zur Verfügung stehen.

\section*{Aufgabe 3}

\end{document}
