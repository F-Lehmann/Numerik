% header
\documentclass[10pt,a4paper]{article}

\usepackage[utf8]{inputenc}
\usepackage{hyperref}
\usepackage{amssymb}
\usepackage{ngerman}
\usepackage{listings}
\usepackage{amsmath}
\usepackage{dsfont}

% the document
\begin{document}

% create the title
% Please replace the data in brackets [] with actual data.
\title{Abgabe - Übungsblatt [$7$]\\
\small{Angewandte Mathematik: Numerik}}
\author{ [Felix Lehmann] \and [Markus Menke]}
\date{\today}
\maketitle

\section*{Aufgabe 1}


\section*{Aufgabe 2}
\subsection*{a)}
Es gibt $2^{53-24}-1=2^{29}-1=536870911$ Zahlen z mit doppelter Genauigkeit ziwschen zwei aufeinanderfolgenden Zahlen x und y einfacher Genauigkeit.
\subsection*{b)}
Die kleinste Natürliche Zahl die sich mit einefacher Genauigkeit nicht ohne Rundungsfehler darstellen lässt ist $2^{24}+1 = 16777217$ da nur 24 Mantissen Bits zur Verfügung stehen.

\section*{Aufgabe 3}

\end{document}
