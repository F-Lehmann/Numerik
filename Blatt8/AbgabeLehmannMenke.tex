% header
\documentclass[10pt,a4paper]{article}

\usepackage[utf8]{inputenc}
\usepackage{hyperref}
\usepackage{amssymb}
\usepackage{ngerman}
\usepackage{listings}
\usepackage{amsmath}
\usepackage{dsfont}

% the document
\begin{document}

% create the title
% Please replace the data in brackets [] with actual data.
\title{Abgabe - Übungsblatt [$8$]\\
\small{Angewandte Mathematik: Numerik}}
\author{ [Felix Lehmann] \and [Markus Menke]}
\date{\today}
\maketitle

\section*{Aufgabe 1}
\subsection*{a)}
$ U = \begin{pmatrix}
    2.3 & 1.8 & 1.0\\
    1.4 & 1.1 & -0.7\\
    0.8 & 4.3 & 2.1\\
\end{pmatrix}
L = \begin{pmatrix}
    1 & 0 & 0\\
    0 & 1 & 0\\
    0 & 0 & 1\\
\end{pmatrix}\\$
Zeilen sind mit römischen Ziffern aufsteigend zu interpretieren:\\
$II = II - (10/23 *  14/10)I\\
U = \begin{pmatrix}
    23/10 & 9/5 & 1\\
    0 & 1/230 & -301/230\\
    4/5 & 43/10 & 21/10\\
\end{pmatrix}
L = \begin{pmatrix}
    1 & 0 & 0\\
    14/23 & 1 & 0\\
    0 & 0 & 1\\
\end{pmatrix}\\
III = III - (10/23 * 8/10)I\\
U = \begin{pmatrix}
    23/10 & 9/5 & 1\\
    0 & 1/230 & -301/230\\
    0 & 169/46 & 403/230\\
\end{pmatrix}
L = \begin{pmatrix}
    1 & 0 & 0\\
    14/23 & 1 & 0\\
    8/23 & 0 & 1\\
\end{pmatrix}\\
III = III - (230 * 169/46)II\\
U = \begin{pmatrix}
    23/10 & 9/5 & 1\\
    0 & 1/230 & -301/230\\
    0 & 0 & 5538/5\\
\end{pmatrix}
L = \begin{pmatrix}
    1 & 0 & 0\\
    14/23 & 1 & 0\\
    8/23 & 845 & 1\\
\end{pmatrix}\\
$
L ist nun untere Dreiecksmatrix und U ist obere Dreiecksmatrix. Der Ergebnis entspricht dem erwarteten Verhalten.\\
Wir errechnen nun die Lösung des LGS:\\
$ L * ( U * x ) = b\\$
Vorwärtseinsetzen (y = Ux):\\
$\begin{pmatrix}
    1 & 0 & 0\\
    14/23 & 1 & 0\\
    8/23 & 845 & 1\\
\end{pmatrix}*\begin{pmatrix}
    y1\\
    y2\\
    y3\\
\end{pmatrix}=\begin{pmatrix}
    1.2\\
    -2.1\\
    0.6\\
\end{pmatrix}\\\Rightarrow
y =\begin{pmatrix}
    6/5\\
    -651/230\\
    23919/10\\
\end{pmatrix} 
$
Jetzt lösen wir das verbleibende LGS:\\
$U * x = y$\\
Welches sich durch Rückwärtseinsetzen lösen lässt:\\
$ x = \begin{pmatrix}
    323/923\\
    -3619/3692\\
    7973/3691\\
\end{pmatrix}$ \\
Da ich die Aufgabe nicht richtig gelesen habe und genau gerechnet habe, ist der relative und absolute Fehler = 0. Ich vermute an dieser Stelle, dass der Fehler recht hoch ist, da die LR Zerlegung numerisch instabil ist. Der relative Fehler wäre an dieser Stelle durch $(|x_{gegeben}-x_{berechnet}|/x_{gegeben}$ berechnet worden. 
\subsection*{b)}
$ U = \begin{pmatrix}
    2.3 & 1.8 & 1.0\\
    1.4 & 1.1 & -0.7\\
    0.8 & 4.3 & 2.1\\
\end{pmatrix}
L = \begin{pmatrix}
    1 & 0 & 0\\
    0 & 1 & 0\\
    0 & 0 & 1\\
\end{pmatrix}\\
P = \begin{pmatrix}
    1 & 0 & 0\\
    0 & 1 & 0\\
    0 & 0 & 1\\
\end{pmatrix}\\$
Maximales $ |u_{i1}| = 2.3 \Rightarrow $ Keine Zeile muss getauscht werden.\\
Maximales $|u_{i2}| = 4.3 \Rightarrow $ Tausche Zeilen II und III in U und P.\\
$ U = \begin{pmatrix}
    2.3 & 1.8 & 1.0\\
    0.8 & 4.3 & 2.1\\
    1.4 & 1.1 & -0.7\\
\end{pmatrix}\\
P = \begin{pmatrix}
    1 & 0 & 0\\
    0 & 0 & 1\\
    0 & 1 & 0\\
\end{pmatrix}\\$
Jetzt lösen wir:
$P*A = L*U$\\
Analog zu Teilaufgabe b) können wir nun L und U berechnen.\\
0en erste Spalte:\\
$U = \begin{pmatrix}
    2.3 & 1.8 & 1\\
    0 & 3.67 & 1.75\\
    0 & 0 & -1.31\\
\end{pmatrix}
L = \begin{pmatrix}
    1 & 0 & 0\\
    0.34 & 1 & 0\\
    0.61 & 0 & 1\\
\end{pmatrix}\\
$
U ist bereits eine obere Dreiecksmatrix\\
Wir lösen $P*L*U*x = b$\\
Sei $y = Ux$\\
Durch Vorwärtseinsetzen erhalten wir y:\\
$y = \begin{pmatrix}
    1.2\\
    -2.51\\
    -0.13\\
\end{pmatrix}\\
$Durch Rückwärtseinsetzen erhalten wir nun x:\\
$x = \begin{pmatrix}
    1.05\\
    -0.73\\
    0.1
\end{pmatrix}$\\
Jetzt muss $P*L*U*x~b$ sein\\
Da ich mich irgendwo massiv verrechnet habe stimmt das leider nicht ganz....\\
Der relative Fehler sollte bei Aufgabeteil b deutlich kleiner sein. Grund dafür ist, dass wir nach weniger Schritten (d.h. auch weniger Rundungen) bereits zum Ergebnis kommen, da U schneller in eine obere Dreiecksmatrix umgewandelt werden kann.\\

\section*{Aufgabe 2}
\lstinputlisting[language=Python, breaklines=true, frame=single]{FrameworkLU.py}

\section*{Aufgabe 3}
\lstinputlisting[language=Python, breaklines=true, frame=single]{FrameworkCholesky.py}

\end{document}
