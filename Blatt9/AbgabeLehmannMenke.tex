% header
\documentclass[10pt,a4paper]{article}

\usepackage[utf8]{inputenc}
\usepackage{hyperref}
\usepackage{amssymb}
\usepackage{ngerman}
\usepackage{listings}
\usepackage{amsmath}
\usepackage{dsfont}

% the document
\begin{document}

% create the title
% Please replace the data in brackets [] with actual data.
\title{Abgabe - Übungsblatt [$9$]\\
\small{Angewandte Mathematik: Numerik}}
\author{ [Felix Lehmann] \and [Markus Menke]}
\date{\today}
\maketitle

\section*{Aufgabe 1}

Cholesky
\begin{enumerate}
    \item Berechne Cholesky Zerlegung von A
    \item Löse unteres Dreieckssystem $R^*w = A^*b$ nach $w$
    \item Löse oberes Dreieckssystem $Rx = w$ nach $x$
\end{enumerate}
Instabil, nur für kleine Probleme\\
nur für hermitesche positiv-definite Matrizen

QR
\begin{enumerate}
    \item Berechne QR Zerlegung von A
    \item Berechne $Q^* b$
    \item Löse oberes Dreieckssystem $R x = Q^* b$ nach $x$
\end{enumerate}
Stabiler, Standardverfahren

SVD
\begin{enumerate}
    \item Berechne SVD Zerlegung von A
    \item Berechne $U^* b$
    \item Löse Diagonalsystem $\Sigma w = U^* b$ nach $w$
    \item Berechne $x = V w$
\end{enumerate}
Ähnliche effizient wie QR wenn $m>>n$, sonst teurer aber auch noch stabiler

\section*{Aufgabe 2}
\subsection*{a)}
$\left(\begin{matrix}
            1 & 2 & 1 \\
            2 & 8 & 2 \\
            1 & 2 & 2
        \end{matrix}\right)
    =\left(\begin{matrix}
            l_{1,1} & 0       & 0       \\
            l_{2,1} & l_{2,2} & 0       \\
            l_{3,1} & l_{3,2} & l_{3,3}
        \end{matrix}\right)
    \cdot\left(\begin{matrix}
            l_{1,1} & l_{2,1} & l_{3,1} \\
            0       & l_{2,2} & l_{3,2} \\
            0       & 0       & l_{3,3}
        \end{matrix}\right)
    =\left(\begin{matrix}
            l_{1,1}^2      & l_{1,1}l_{2,1}                & l_{1,1}l_{3,1}                \\
            l_{1,1}l_{2,1} & l_{2,1}^2+l_{2,2}^2           & l_{2,1}l_{3,1}+l_{2,2}l_{3,2} \\
            l_{1,1}l_{3,1} & l_{2,1}l_{3,1}+l_{2,2}l_{3,2} & l_{3,1}^2+l_{3,2}^2+l_{3,3}^2
        \end{matrix}\right)$\\
\begin{align*}
    l_{1,1}^2 = 1                     & \rightarrow l_{1,1} = 1 \\
    l_{1,1}l_{2,1} = 2                & \rightarrow l_{2,1} = 2 \\
    l_{1,1}l_{3,1} = 1                & \rightarrow l_{3,1} = 1 \\
    l_{2,1}^2+l_{2,2}^2 = 8           & \rightarrow l_{2,2} = 2 \\
    l_{2,1}l_{3,1}+l_{2,2}l_{3,2} = 2 & \rightarrow l_{3,2} = 0 \\
    l_{3,1}^2+l_{3,2}^2+l_{3,3}^2 = 2 & \rightarrow l_{3,3} = 1
\end{align*}
$L=\left(\begin{matrix}
            1 & 0 & 0 \\
            2 & 2 & 0 \\
            1 & 0 & 1
        \end{matrix}\right)$
$R=\left(\begin{matrix}
            1 & 2 & 1 \\
            0 & 2 & 0 \\
            0 & 0 & 1
        \end{matrix}\right)$
\subsection*{b)}
\begin{align*}
    R^*w &= b\\
    \left(\begin{matrix}
        1 & 0 & 0 \\
        2 & 2 & 0 \\
        1 & 0 & 1
    \end{matrix}\right) w &=  \left(\begin{matrix}
        4 \\
        12 \\
        8
    \end{matrix}\right)\\
    \Rightarrow w &= \left(\begin{matrix}
        4 \\
        2 \\
        4
    \end{matrix}\right)
\end{align*}
\begin{align*}
    Rx &= w\\
    \left(\begin{matrix}
        1 & 2 & 1 \\
        0 & 2 & 0 \\
        0 & 0 & 1
    \end{matrix}\right) x &=  \left(\begin{matrix}
        4 \\
        2 \\
        4
    \end{matrix}\right)\\
    \Rightarrow x &= \left(\begin{matrix}
        -2 \\
        1 \\
        4
    \end{matrix}\right)
\end{align*}

\section*{Aufgabe 3}
\lstinputlisting[language=Python, breaklines=true, frame=single]{FrameworkEigenvalue.py}

\end{document}
